\chapter{Graph}

\section{Fundamentals}
	% \kactlimport{BellmanFord.h}
	\kactlimport{SPFA.h}
	% \kactlimport{FloydWarshall.h}
	\kactlimport{TopoSort.h}

\section{Network flow}
	% \kactlimport{PushRelabel.h}
	\kactlimport{Dinic.h}
	\kactlimport{MinCostMaxFlow.h}
	\kactlimport{MCMF-OH.h}
	% \kactlimport{EdmondsKarp.h}
	\kactlimport{MinCut.h}
	\kactlimport{GlobalMinCut.h}
	% \kactlimport{GomoryHu.h}

\section{Matching}
	\kactlimport{hopcroftKarp.h}
	% \kactlimport{DFSMatching.h}
	% \kactlimport{MinimumVertexCover.h}
	\kactlimport{DFSMatching-PO.h}
	\kactlimport{WeightedMatching.h}
	% \kactlimport{GeneralMatching.h}
	\subsection{Hall's Theorem}
		Let $G$ be a finite bipartite graph with bipartite sets $X$ and $Y$ (i.e. $G := (X+Y, E)$). 
		An $X$-perfect matching (also called: $X$-saturating matching) is a matching which covers every vertex in $X$. 
		For a subset $W$ of $X$, let $N_{G}(W)$ denote the neighborhood of $W$ in $G$, i.e., the set of all vertices in $Y$ adjacent to some element of $W$. 
		The marriage theorem in this formulation states that there is an $X$-perfect matching if and only if for every subset $W$ of $X$:
		$$|W|\le{}|N_{G}(W)|$$
		In other words: every subset $W$ of $X$ has sufficiently many adjacent vertices in $Y$.

\section{DFS algorithms}
	\kactlimport{SCC.h}
	% \kactlimport{BiconnectedComponents.h}
	\kactlimport{BCC.h}
	\kactlimport{2sat.h}
	\kactlimport{EulerWalk.h}

\section{Coloring}
	% \kactlimport{EdgeColoring.h}

% \section{Heuristics}
% 	\kactlimport{MaximalCliques.h}
% 	\kactlimport{MaximumClique.h}
% 	\kactlimport{MaximumIndependentSet.h}

\section{Trees}
	\kactlimport{BinaryLifting.h}
	% \kactlimport{LCA.h}
	% \kactlimport{CompressTree.h}
	\kactlimport{HLD.h}
	\kactlimport{Centroid.h}
	% \kactlimport{LinkCutTree.h}
	% \kactlimport{DirectedMST.h}
	\kactlimport{ManhattanMST.h}

\section{Math}
	\subsection{Number of Spanning Trees}
		% I.e. matrix-tree theorem.
		% Source: https://en.wikipedia.org/wiki/Kirchhoff%27s_theorem
		% Test: stress-tests/graph/matrix-tree.cpp
		Create an $N\times N$ matrix \texttt{mat}, and for each edge $a \rightarrow b \in G$, do
		\texttt{mat[a][b]--, mat[b][b]++} (and \texttt{mat[b][a]--, mat[a][a]++} if $G$ is undirected).
		Remove the $i$th row and column and take the determinant; this yields the number of directed spanning trees rooted at $i$
		(if $G$ is undirected, remove any row/column).
		$$$$

	\subsection{Erdős–Gallai theorem}
		Source: https://en.wikipedia.org/wiki/Erd%C5%91s%E2%80%93Gallai_theorem
		Test: stress-tests/graph/erdos-gallai.cpp
		A simple graph with node degrees $d_1 \ge \dots \ge d_n$ exists iff $d_1 + \dots + d_n$ is even and for every $k = 1\dots n$,
		\[ \sum _{i=1}^{k}d_{i}\leq k(k-1)+\sum _{i=k+1}^{n}\min(d_{i},k). \]
