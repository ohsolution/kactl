\chapter{Graph}

\section{Fundamentals}
	% \kactlimport{BellmanFord.h}
	\kactlimport{SPFA.h}
	% \kactlimport{FloydWarshall.h}
	\kactlimport{TopoSort.h}

\section{Network flow}
	% \kactlimport{PushRelabel.h}
	\kactlimport{Dinic.h}
	\kactlimport{MinCostMaxFlow.h}
	\kactlimport{MCMF-OH.h}
	% \kactlimport{EdmondsKarp.h}
	\kactlimport{MinCut.h}
	\kactlimport{GlobalMinCut.h}
	% \kactlimport{GomoryHu.h}

	\section{Matching}
	\subsection{L-R max flow}
		1. Add new source and sink, $S'$ and $T'$ (note that, original source and sink are $S$ and $T$) \\
		2. Change given edge's capacity to $R-L$ (where lower bound$=L$, upper bound$=R$, direction is $v_1$ to $v_2$) \\
		3. Add new edge from $S'$ to $v_2$, and from $v_1$ to $T'$ and its capacity is $L$ \\
		4. Add new edge from $T$ to $S$ with capacity infinity. \\
		5. To check its validity, execute flow algoritm from $S'$ to $T'$. If maximum flow from $S'$ to $T'$ is equal to sum of lower bound capacity, there exists solution under the constraint. \\
		6. To find maximum flow, execute flow algorithm from original source to original sink. That is, from $S$ to $T$. \\
		% lower bound = $L$, upper bound = $R$, change edge from $v_1$ to $v_2$ as capacity $= R-L$. Add new Source 
		% \hspace{1em}L-R max flow는 Flow Graph 중 각 edge의 유량에 대한 하한선과 상한선이 존재하는 경우에 대한 최대 유량을 찾는 방법이다. 
		% 모델링 방법은 기존의 Flow Graph에서 새로운 Source와 새로운 Sink를 추가하는 것이다. 새로운 Source와 Sink에 edge를 연결하는 방법은 다음과 같다. 
		% 정점 v1에서 v2로 하한유량 L과 상한유량 R을 갖는 edge가 있다고 가정하자. 
		% 새로운 Source에서 v2로 연결되는 capacity L짜리 edge를 연결하고, v1에서 새로운 Sink로 capacity가 L인 edge를 새롭게 연결하면 된다. 
		% 이후 마지막으로 기존의 Sink에서 기존의 Source로 가는 capacity가 무한대인 edge를 추가한 뒤, 기존의 그래프의 edge 중 상한과 하한을 갖는 edge들의 capacity를 R-L로 수정한다. 
		% 이렇게하면 L-R flow를 구하기 위한 새로운 그래프가 완성된다. 
		% 이제 조건을 만족하는 정답이 존재하는지 판별하는 방법은 새로운 Source에서 새로운 Sink로 들어오는 유량이 $L_{tot}$인지 확인하면되고, 
		% 만약 존재성이 밝혀진다면, 현재의 Flow Graph의 상태를 유지한채 기존의 Source에서 Sink로 가는 Maximum Flow를 구하여 존재성이 밝혀졌을 때의 $f_{Sink,Source}$에 더해주면 구하고자 하는 정답이 된다.
	\subsection{Hall's Theorem}
		Let $G$ be a finite bipartite graph with bipartite sets $X$ and $Y$ (i.e. $G := (X+Y, E)$). 
		An $X$-perfect matching (also called: $X$-saturating matching) is a matching which covers every vertex in $X$. 
		For a subset $W$ of $X$, let $N_{G}(W)$ denote the neighborhood of $W$ in $G$, i.e., the set of all vertices in $Y$ adjacent to some element of $W$. 
		The marriage theorem in this formulation states that there is an $X$-perfect matching if and only if for every subset $W$ of $X$:
		$$|W|\le{}|N_{G}(W)|$$
		In other words: every subset $W$ of $X$ has sufficiently many adjacent vertices in $Y$.
	\kactlimport{hopcroftKarp.h}
	% \kactlimport{DFSMatching.h}
	% \kactlimport{MinimumVertexCover.h}
	\kactlimport{DFSMatching-PO.h}
	\kactlimport{WeightedMatching.h}
	% \kactlimport{GeneralMatching.h}


\section{DFS algorithms}
	\kactlimport{SCC.h}
	% \kactlimport{BiconnectedComponents.h}
	\kactlimport{BCC.h}
	\kactlimport{2sat.h}
	\kactlimport{EulerWalk.h}

\section{Coloring}
	% \kactlimport{EdgeColoring.h}

% \section{Heuristics}
% 	\kactlimport{MaximalCliques.h}
% 	\kactlimport{MaximumClique.h}
% 	\kactlimport{MaximumIndependentSet.h}

\section{Trees}
	\kactlimport{BinaryLifting.h}
	% \kactlimport{LCA.h}
	% \kactlimport{CompressTree.h}
	\kactlimport{HLD.h}
	\kactlimport{Centroid.h}
	% \kactlimport{LinkCutTree.h}
	% \kactlimport{DirectedMST.h}
	\kactlimport{ManhattanMST.h}

\section{Math}
	\subsection{Number of Spanning Trees}
		% I.e. matrix-tree theorem.
		% Source: https://en.wikipedia.org/wiki/Kirchhoff%27s_theorem
		% Test: stress-tests/graph/matrix-tree.cpp
		Create an $N\times N$ matrix \texttt{mat}, and for each edge $a \rightarrow b \in G$, do
		\texttt{mat[a][b]--, mat[b][b]++} (and \texttt{mat[b][a]--, mat[a][a]++} if $G$ is undirected).
		Remove the $i$th row and column and take the determinant; this yields the number of directed spanning trees rooted at $i$
		(if $G$ is undirected, remove any row/column).
		$$$$

	\subsection{Erdős–Gallai theorem}
		Source: https://en.wikipedia.org/wiki/Erd%C5%91s%E2%80%93Gallai_theorem
		Test: stress-tests/graph/erdos-gallai.cpp
		A simple graph with node degrees $d_1 \ge \dots \ge d_n$ exists iff $d_1 + \dots + d_n$ is even and for every $k = 1\dots n$,
		\[ \sum _{i=1}^{k}d_{i}\leq k(k-1)+\sum _{i=k+1}^{n}\min(d_{i},k). \]
